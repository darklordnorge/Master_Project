%\addcontentsline{toc}{chapter}{Development Process}
\chapter{Design}



\section{Overall Architecture}
\subsection{Control Algorithm}
In this section the design of the control algorithm will be explained.

\subsection{Artificial Neural Network}
The robot swarm is controlled by an artificial neural network. \\
The neural network is consistent of 8 inputs, 3 hidden nodes, and 4 outputs. \\
The inputs are taken from the E-Pucks 8 IR proximity sensors. The hidden layer consists of 3 hidden nodes, which give more computational depth to the network. \\
From the 4 outputs of the neural the speed of the 2 stepper motors of the e-puck are calculated. This is done by calculating the difference between them. \\
The weights of the neural network are generated by a Genetic algorithm, which is part of the used simulator. The genes created by the GA are a value between 0 and 1, however the neural network algorithm scales them to be be a value between -5 and 5.\\ 
%explain reason for scalling
%use geogebra to draw a sigmoid function and explain  creater activation function spread 

There are bias nodes connected to the hidden and output layer, both of which are always set to 1. \\
The bias is needed to be able to shift the entire sigmoid function along the \textit{x} axis. 

\begin{figure}[h]
\begin{center}
\includegraphics[scale=0.3]{Chapter2/images/bias.png}
\caption[Representation of a shifting sigmoid function]{Representation of a shifting sigmoid function\footnotemark}
\label{fig:bias}
\end{center}
\end{figure}

\footnotetext{Image credit Stackoverflow, accessed 8th of September, 2015 \url{http://goo.gl/Vktx0X}}



The ANN is a multilayer feed forward network, meaning all layers nodes are connected to all nodes in the following layer and that data is only passed forward in the network, never back as would be the case using a different type of neural network, like a backpropagation algorithm. \\



\subsubsection{Input Layer}
The inputs to the ANN are returned from the E-Pucks 8 IR sensors and are a value between 0 and 4096.\\

\begin{figure}[h]
\begin{center}
\includegraphics[scale=0.3]{Chapter2/images/Selection_002.png} 
\caption[IR sensor response against distance]{IR sensor response against distance\footnotemark}
\label{fig:ir_distance}
\end{center}
\end{figure} 

\footnotetext{Image credit: Webots User Guide \url{http://goo.gl/kyCINM}}

As can be in the in figure \ref{fig:ir_distance} the returned IR sensors value rises drastically after the robot comes closer to an obstacle than 3 centimetres. \\


\subsubsection{Hidden Layer \& Sigmoid Function}
In the hidden layer the sum of all inputs to a node is multiplied by the weights to that node and than fed to sigmoid function.\\

A sigmoid function refers to a mathematical function that has an "S"(sigmoid) shape.\\
A sigmoid, or activation function, is an abstract representation of a neuron firing(activating) in the brain. There are a number of different approaches to activation functions, the simplest is a simple binary step function, with only 2 stages: \textit{on} or \textit{off}.\\
The activation function in this neural network is a sigmoid function which is given by:

\begin{equation}
S(x) = \frac{1}{1 + e^{-x}}
\end{equation} 

Where $e$ represents \textit{Euler's number} which is 2.71828[...] and $x$ represents the input to the function, in this case the sum of all inputs multiplied by the weights.\\
The output of the sigmoid function is a number between 0 and 1.\\




\subsubsection{Output Layer}


\subsection{Fitness Function}\label{chap2:fitness_function}

\section{Some detailed design}

\subsection{Even more detail}

\section{User Interface}

\section{Other relevant sections}