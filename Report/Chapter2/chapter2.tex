%\addcontentsline{toc}{chapter}{Development Process}
\chapter{Design}



\section{Overall Architecture}
\subsection{Control Algorithm}
In this section the design of the control algorithm will be explained.

\subsection{Artificial Neural Network}
The robot swarm is controlled by an artificial neural network. \\
The neural network is consistent of 8 inputs, 3 hidden nodes, and 4 outputs. \\
The inputs are taken from the E-Pucks 8 IR proximity sensors. The hidden layer consists of 3 hidden nodes, which give more computational depth to the network. \\
From the 4 outputs of the neural the speed of the 2 stepper motors of the e-puck are calculated. This is done by calculating the difference between them. \\
The weights of the neural network are generated by a Genetic algorithm, which is part of the used simulator. 
There are bias nodes connected to the hidden and output layer, both of which are always set to 1. \\

The inputs to the ANN are returned from the E-Pucks 8 IR sensors and are a value between 0 and 4096.\\

\begin{figure}[h]
\begin{center}
\includegraphics[scale=0.3]{Chapter2/images/Selection_002.png} 
\caption[IR sensor response against distance]{IR sensor response against distance\footnotemark}
\label{fig:ir_distance}
\end{center}
\end{figure} 

\footnotetext{Image credit: Webots User Guide \url{http://goo.gl/kyCINM}}

As can be in the in figure \ref{fig:ir_distance} the returned IR sensors value rises drastically after the robot comes closer to an obstacle than 3 centimetres. 

\section{Some detailed design}

\subsection{Even more detail}

\section{User Interface}

\section{Other relevant sections}