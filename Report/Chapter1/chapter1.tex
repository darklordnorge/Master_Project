\chapter{Background \& Objectives}

This section should discuss your preparation for the project, including background reading, your analysis of the problem and the process or method you have followed to help structure your work.  It is likely that you will reuse part of your outline project specification, but at this point in the project you should have more to talk about. 

\textbf{Note}: 

\begin{itemize}
   \item All of the sections and text in this example are for illustration purposes. The main Chapters are a good starting point, but the content and actual sections that you include are likely to be different.
   
   \item Look at the document on the Structure of the Final Report for additional guidance. 
   
\end {itemize}

\section{Background}
What was your background preparation for the project? What similar systems did you assess? What was your motivation and interest in this project? 

\section{Analysis}
Taking into account the problem and what you learned from the background work, what was your analysis of the problem? How did your analysis help to decompose the problem into the main tasks that you would undertake? Were there alternative approaches? Why did you choose one approach compared to the alternatives? 

There should be a clear statement of the objectives of the work, which you will evaluate at the end of the work. 

In most cases, the agreed objectives or requirements will be the result of a compromise between what would ideally have been produced and what was felt to be possible in the time available. A discussion of the process of arriving at the final list is usually appropriate.

\subsection{Communications}
The Communication capabilities of the E-Puck where analysed. The Standard E-Puck comes  with bluetooth communication and posses now WiFi capabilities. \\
Bluetooth  communication for this project has been deemed infeasible as Bluetooth communications can take somewhere around 19.5 $\pm$ 4 seconds. A multi robot exploration and mapping project such as this requires almost constant communication, in which case bluetooth connection times of \~19 seconds are too long.\\

There are a few proposed ways to implement WiFi communications on the e-puck robot. 
One of the methods was proposed by Christopher M. Cianci \textit{et al.}\cite{Cianci2007Communication} is the creation and implementation of a WiFi extension board for the e-puck, enabling communication between ZigBee and other IEEE 802.15.4 compliant transceivers. \\
There designed communication board is based on the MSP430 Microcontroller\footnote{\url{http://www.ti.com/product/msp430f169}} and the Chipcon CC2420\footnote{http://www.ti.com/product/cc2420} radio.\\
Allowing the e-puck a communication range between 15cm and 5 meters. \\
However such an board is not commercially available and would need to be custom designed and build, which is outside the spectrum of this project.

For the purposes of this project the use of the official e-puck range and bearing board has been deemed appropriate, as it is would be commercially available.\\
The range and bearing board is an extension board for the E-Puck which allows for localisation and local communication between E-Pucks using infra-red transmission. 
The board is powered by its own processor and consists of 12 sets of IR emission/reception modules. 
The board was first designed and build by Guiêrrez \textit{et al.}\cite{Gutierrez}. 

\section{Process}
You need to describe briefly the life cycle model or research method that you used. You do not need to write about all of the different process models that you are aware of. Focus on the process model that you have used. It is possible that you needed to adapt an existing process model to suit your project; clearly identify what you used and how you adapted it for your needs.\\



